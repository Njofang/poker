\documentclass[12pt,a4paper]{article}
\usepackage[utf8]{inputenc}
\usepackage[french]{babel}
\usepackage[T1]{fontenc}
\usepackage{graphicx}

\begin{document}
\section{Initialisation}
\renewcommand\labelitemi{$\bullet$}
\begin{itemize}
\item{
Déclare un jeu de 52 cartes qui sera modélisé à l'aide de structures pour identifier la couleur et la hauteur des cartes.}
\item{Création des joueurs caractérisé par un pseudonyme, un nombre de jetons, et une main d'un nombre constant de 5 cartes.}
\end{itemize}

\section{Mélange}
\begin{itemize}
\item{
Mélange le jeu en affectant une valeur aléatoire à chaque carte }
\end{itemize}

\section{Distribution}
\begin{itemize}
\item{
Distribue 5 cartes de la pioche à chaque joueur}
\end{itemize}

\section{Tour de jeu}
\begin{itemize}
\item{Chaque tour est caractérisé par 2 phases de tirage\footnote{phase du jeu qui permet à chaque joueur d'échanger n'importe quelle carte(s) de sa main avec la pioche} et 3 phases de mise}
\item{Sens horaire à gauche du croupier\footnote{joueur désigné pour distribuer les cartes} de distribution des cartes et du tour des joueurs }
\item{Le premier joueur doit miser la "petite blind" et le deuxième la "big blind".}
\subsection{ Déroulement phase de mise}
\begin{description}
\item[Début d'une mise : ]{Le joueur après la big blind doit miser la somme minimal  de la big blind et miser au maximum son tapis }
\item[Suite : ]{Le tour de mise se poursuit jusqu'à que tous les joueurs aient misé le même nombre de jeton}
\end{description}
\subsection{Tirage}
Les joueurs échangent à tour de rôle, dans le même sens de jeu de la mise, un nombre de carte compris entre 0 et 4 au maximum.
\end{itemize}
\section{Abattage}
Les joueurs présentent leurs cartes et le joueur avec la meilleure combinaison remporte le pot. En cas d'égalité, le pot est partagé et réparti équitablement entre les joueurs à égalité.
\section{Répartition des tâches}
\subsection{Version 1}
\begin{description}
\item[Romain : ]{Création et initialisation des objets (carte, jeu de carte, joueur, tapis de jeu), interactions des cartes entres les joueurs et le jeu de cartes, couche graphique en SDL}
\item[Tristan : ]{Abattage : analyse des combinaison des mains des joueurs, détermination du gagnant, et répartition des gains}
\item[William :]  {Tour de jeu : système de mise et tirage}
\item
\item[OPTIONEL : ]{Jeu multijoueur en réseau local}

\subsection{Version 2}
\item[Romain : Phase 1]
{
\begin{itemize}
\item création de l'architecture de base du jeu (modules, bibliothèques et makefile)
\item création des structures correspondant aux cartes, au jeu de cartes et au joueur
\item implémentation des primitives de base des structures (création, initialisation et destruction)
\end{itemize}
}
\item[Romain : phase 2]
{
\begin{itemize}
\item Changement du système de représentation et de la structure du jeu de cartes : tableau de pointeurs sur structures de cartes vers une liste
\item Abandon et retour vers un tableau de pointeurs sur des cartes
\item implémentation de nouvelles primitives de modification et d'interaction entre les structures
\end{itemize}
}
\item[Romain : phase 3]
{
\begin{itemize}
\item implémentation d'un ecran de sélection du mode de l'écran (plein écran ou fenetre) et du menu principal en sdl 
\item ajout et gestion des effets sonores via un système de canaux audio
\item ajout de la fenêtre de jeu (affichage table de poker et cartes recto/verso)
\item ajout d'un assistant graphique d'installation/désinstallation du jeu
\item amélioration du makefile (ajout de cibles pour installer/désinstaller sdl, menu help, ...)
\end{itemize}
}
\item[Romain : phase 4]
{
\begin{itemize}
\item séparation du programme principal du code de manipulation de m'interface graphique : création des fonctions de gestion de l'interface grpahique dans les nouveaux fichiers gui.c et gui.h
\end{itemize}
}

\item[Tristan :]
{
Fait
\begin{itemize}
    \item Abbatage : Analyse des combinaisons des mains des joueurs
    \item Détermination du gagnant
\end{itemize}
A faire
\begin{itemize}
    \item Assiste William sur système de mise et tirage
    \item Multijoueur : Création d'un serveur acceuillant 5 clients maximum
\end{itemize}
}
\item[William :]{Tour de jeu: système de mise et tirage, répartition des gains}

\end{description}

\section{Logigramme provisoire}
\includegraphics[scale=0.69]{logigramme}

\end{document}
